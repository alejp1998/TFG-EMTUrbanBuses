\documentclass{beamer}

%%%%%%%%%%%%%%%%%%%%%%%%
% Usual LaTeX Packages %
%%%%%%%%%%%%%%%%%%%%%%%%

\usepackage[spanish]{babel}
\usepackage[utf8]{inputenc} %Codificacion utf-8

\usepackage{amsmath}
\usepackage{amsfonts}
\usepackage{amssymb}
\usepackage{mathrsfs} 			% For Weinberg-esque letters
\usepackage{cancel}				% For "SUSY-breaking" symbol
\usepackage{slashed}            % for slashed characters in math mode
\usepackage{bbm}                % for \mathbbm{1} (unit matrix)
\usepackage{amsthm}				% For theorem environment
\usepackage{multirow}			% For multi row cells in table
\usepackage{arydshln} 			% For dashed lines in arrays and tables

\usepackage[T1]{fontenc}
\usepackage{tikz}
\usetikzlibrary{arrows,shapes}


\graphicspath{{images/}}	% Put all images in this directory. Avoids clutter.

\usetheme[progressbar=frametitle]{metropolis}
\usepackage{appendixnumberbeamer}

\usepackage{booktabs}
\usepackage[scale=2]{ccicons}

\usepackage{pgfplots}
\usepgfplotslibrary{dateplot}

\usepackage{xcolor}
\usepackage{xspace}
\newcommand{\themename}{\textbf{\textsc{metropolis}}\xspace}

% To make comments
\usepackage{todonotes}
\usepackage{soul}
\usepackage{hyperref}
\newcommand{\tb}{\textcolor{blue}}
\newcommand{\tr}{\textcolor{red}}
\colorlet{RED}{red}
\newcommand{\tgr}{\textcolor{lightgreen}}
\colorlet{RED}{red}
\newcommand{\tod}{\todo[color=bluep]}
\newcommand{\todi}{\todo[inline,color=bluep]}
\newcommand{\tody}{\todo[inline,color=yellow]}
\definecolor{bluep}{RGB}{3, 252, 244}
\definecolor{lightgreen}{HTML}{28B463}


\newcommand{\titulo}{Análisis estadístico de tiempos de llegada y detección de anomalías en los buses urbanos de la EMT}
\newcommand{\TITULO}{ANÁLISIS ESTADÍSTICO DE TIEMPOS DE LLEGADA Y DETECCIÓN DE ANOMALÍAS EN LOS AUTOBUSES URBANOS DE LA EMT}
\newcommand{\titlee}{Statistical arrival time analysis and anomaly detection in the EMT urban buses}
\newcommand{\me}{Alejandro Jarabo Peñas}
\newcommand{\ME}{ALEJANDRO JARABO PEÑAS}
\newcommand{\tutor}{Pedro José Zufiria Zatarain}
\newcommand{\TUTOR}{PEDRO JOSÉ ZUFIRIA ZATARAIN}


%%%%%%%%%%%%%%%%%%%%%%%%% NOTATION %%%%%%%%%%%%%%%%%%%%%%%%%%%

%Arrival time
\newcommand{\runningTime}{\ensuremath{\Delta T_{d,o,k}}}

%Headway
\newcommand*{\hw}[2]{
    \Delta T_{b_{#1},b_{#2}}
}

%Headway
\newcommand*{\hway}[2]{
    \Delta T_{#1,#2}
}

%Scientifical notation
\newcommand*{\scinot}[2]{
    #1 \cdot 10^{#2}
}
%%%%%%%%%%%%%%%%%%%%%%%%%%%%%%%%%%%%%%%%%%%%%%%%%%%%%%%%%%%%%%

\author{\me}
\title{\titulo}
\institute{ETSIT UPM}
\date{\today}

\mode<presentation>
{
    \usetheme{default}      % or try Darmstadt, Madrid, Warsaw, ...
    \usecolortheme{default} % or try albatross, beaver, crane, ...
    \usefonttheme{default}  % or try serif, structurebold, ...
    \setbeamertemplate{navigation symbols}{}
    \setbeamertemplate{caption}[numbered]
    \setbeamercolor{background canvas}{bg=white}
    \setbeamercolor{progress bar}{%
      use=alerted text,
      fg=purple,
      bg=alerted text.fg!50!black!30
    }
} 

\begin{document}
\tikzstyle{every picture}+=[remember picture]

{\setbeamertemplate{sidebar right}{\llap{
%Image commented for a faster compilation
\includegraphics[width=\paperwidth,height=0.7\paperheight]{images/busemt.png}
}}
\begin{frame}[c]
\begin{center}
    
    \vspace{2em}
    
    \colorbox{white}{\parbox{\dimexpr\linewidth-2\fboxsep}{\centering\large \textcolor{black}{\textsc{\titulo}}}}
	%\Large \textcolor{white}{\textsc{\title}}
	
	\vspace{12em}
	
	Autor: \me.\\
	Tutor: \tutor.\\
	
	\vspace{0.5em}
	
	\includegraphics[height=1cm]{images/catedracabify.png} \quad
	\includegraphics[height=1.3cm]{images/etsitupm.png}\\
\end{center}
\end{frame}
}

\begin{frame}{Tabla de contenidos}
  \setbeamertemplate{section in toc}[sections numbered]
  \tableofcontents%[hideallsubsections]
\end{frame}


\section{Motivación y resumen}



\begin{frame}{Motivación y resumen}
  \begin{columns}[T,onlytextwidth]
    \column{0.39\textwidth}
        \textsc{Importancia del transporte público:}
        
        \vspace{1em}
    
        \small \tgr{$\downarrow$ Contribución de cada ciudadano a la huella de carbono.}
        
        \vspace{0.5em}
        
        \small \tgr{$\uparrow$ Accesibilidad y facilidad de los desplazamientos por las ciudades.}
        
    \column{0.6\textwidth}
        \textsc{Metodología}
        \centering \metroset{block=fill}
        \begin{block}{\scriptsize Recopilación, análisis y limpieza de los datos}
            \tiny Tiempos de llegada a cada parada proporcionados por la API
        \end{block}
        \scriptsize $\downarrow$
        \begin{block}{\scriptsize Estimación de datos derivados}
            \tiny Tiempos entre paradas, headways, etc.
        \end{block}
        \scriptsize $\downarrow$
        \begin{alertblock}{\scriptsize Detección de anomalías en tiempo real sobre los headways}
            \tiny Series k-Dimensionales de headways y distancia de Mahalanobis
        \end{alertblock}
        \scriptsize $\downarrow$
        \begin{block}{\scriptsize Monitorización de la detección de anomalías}
            \tiny Dashboard con visualizaciones en tiempo real
        \end{block}
        
  \end{columns}
\end{frame} 


\section{Datos disponibles y su adquisición}



\begin{frame}{Datos estáticos}
    \begin{columns}[T,onlytextwidth]
        \metroset{block=fill}
        \column{0.34\textwidth}
            
            \vspace{3em}
            
            \begin{exampleblock}{GTFS \tikz[baseline]\node[coordinate](n1){};}
                \scriptsize 
                Formato de datos estáticos para servicios de transporte público:
                \tiny
                \begin{itemize}
                    \item Rutas
                    \item Paradas 
                    \item Horarios
                    \item Tarifas
                \end{itemize}
            \end{exampleblock}
            
        \column{0.5\textwidth}
        
            \tikz[baseline]\node[coordinate](t1){}; 
            Mapa de líneas y paradas
            \includegraphics[width=0.85\textwidth]{images/emturbanmap.png}
            
            \tikz[baseline]\node[coordinate](t2){}; 
            \begin{block}{\scriptsize Diccionario de las líneas}
                \scriptsize 
                Datos más importantes para cada línea:
                \tiny
                \begin{itemize}
                    \item Longitud total
                    \item Paradas ordenadas con sus distancias al comienzo de la línea 
                    \item Destinos
                \end{itemize}
            \end{block}
        
    \end{columns}
    
    \tikz[overlay]{\path[->]<1-> ([xshift=6em,yshift=-4.1em]n1) edge [bend left] ([xshift=1em,yshift=-1em]t1);}
    \tikz[overlay]{\path[->]<1-> ([xshift=6em,yshift=-6.2em]n1) edge [bend right] ([xshift=0.5em,yshift=-5em]t2);}
\end{frame}


\begin{frame}{Datos en tiempo real}

    \begin{columns}[T]
        \metroset{block=fill}
        \column{0.5\textwidth}
            
            \begin{exampleblock}{MobilityLabs EMT API}
                \scriptsize 
                Peticiones a métodos $\rightarrow$ información en tiempo real:
                \begin{itemize}
                    \item \textbf{TimeArrivalBus(stopId)}: próximos buses que van a llegar a la parada especificada.
                \end{itemize}
            \end{exampleblock}
            \tikz[baseline]\node[coordinate](n1){};
            \centering \tiny
            \textbf{$D^s_b$ $\equiv$ Distancia restante de bus $b$ para llegar a parada $s$}\\
        
            \textbf{$T^s_b$ $\equiv$ Tiempo restante de bus $b$ para llegar a parada $s$}
            \flushleft
            \tikz[baseline]\node[coordinate](t1){};
            \scriptsize Respuesta en formato json 
            \vspace{1em}
            \includegraphics[width=1\textwidth]{images/timearrivalbus.png}
            
        \column{0.5\textwidth}
            \centering
            \begin{alertblock}{Limitación}
                \scriptsize 
                330000 hits diarios
            \end{alertblock}
            {\scriptsize $\downarrow$ Selección de un subconjunto}
            
            \begin{block}{Líneas 1, 44, 82, 132 y 133}
                \tiny
                \begin{itemize}
                    \item 272 paradas distintas.
                    \item Una ráfaga de peticiones a todas las paradas cada 50 segundos ($T_m=50s$), desde las 7am a las 11pm.
                \end{itemize}
                
                \centering
                \includegraphics[width=0.6\textwidth]{images/newselectedlines.png}
            \end{block}
        
    \end{columns}
    
    \tikz[overlay]{\path[->]<1-> ([xshift=0.5em,yshift=3em]n1) edge [bend right] ([xshift=0em,yshift=-0.5em]t1);}
    
\end{frame} 


\section{Análisis y limpieza de los datos}



\begin{frame}{Caracterización de la API y los datos que proporciona}
    \begin{columns}[T]
        \metroset{block=fill}
        \column{0.4\textwidth}
            Línea F con $T_m=5s$
            \begin{block}{\scriptsize Tasa de refresco}
                \tiny Moda cada 5 s; media cada 30 s. 
            \end{block}
            
            \begin{alertblock}{\scriptsize Geolocalización de buses\tikz[baseline]\node[coordinate](n1){};}
                \tiny Próxima a la línea que siguen pero muy distante a su ubicación real.
            \end{alertblock}
            
            \begin{block}{\scriptsize Velocidad del bus}
                \tiny Valores constantes durante aprox. 30 s. Valor máximo 30 km/h.
            \end{block}
        
        \column{0.6\textwidth}
            \centering \scriptsize 
            
            \tikz[baseline]\node[coordinate](t1){};
            Distancia a la línea
            \includegraphics[width=0.8\textwidth]{images/distancetoline.png}
            
            \tikz[baseline]\node[coordinate](t2){};
            Distancia a la parada a la que han llegado
            \includegraphics[width=0.8\textwidth]{images/distancetostop.png}
            
            \tikz[baseline]\node[coordinate](t3){};
            Posición de la parada vs posición del bus
            \includegraphics[width=0.8\textwidth]{images/distancetostoponmap.png}
        
    \end{columns}
    
    \tikz[overlay]{\path[->]<1-> ([xshift=2em,yshift=0.5em]n1) edge [bend left] ([xshift=-3em,yshift=-1em]t1);}
    \tikz[overlay]{\path[->]<1-> ([xshift=2em,yshift=0.5em]n1) edge [bend right] ([xshift=0.5em,yshift=-1em]t2);}
    \tikz[overlay]{\path[->]<1-> ([xshift=2em,yshift=0.5em]n1) edge [bend right] ([xshift=0.5em,yshift=-1em]t3);}
\end{frame}


\begin{frame}{Limpieza de los datos}
    \begin{columns}[T,onlytextwidth]
        \metroset{block=fill}
        \column{0.4\textwidth}
            
            \begin{alertblock}{\scriptsize Datos contaminados}
                \scriptsize Pueden llevar a conclusiones erróneas:
                \begin{itemize}
                    \tiny
                    \item Tiempo restante inusualmente elevado (A).
                    
                    \item Distancia restante menor que 0 (B).
                    
                \end{itemize}
            \end{alertblock}
            
            \vspace{1em}
            A.\includegraphics[width=0.85\textwidth]{images/outlier1.png}
            \\[1em]
            B.\includegraphics[width=0.85\textwidth]{images/outlier2.png}
            
        \column{0.5\textwidth}
            
            \vspace{1em}
            \textbf{Criterios de consistencia}
            
            \scriptsize
            \begin{itemize}
                \item Consistencia entre línea, destino y parada.
                \item Distancias restantes:
                \begin{itemize}
                \tiny
                    \item $>=$ 0
                    \item $<$ longitud de la línea
                \end{itemize}
                
                \item Tiempos restantes:
                \begin{itemize}
                \tiny
                    \item $>=$ 0
                    \item $<$ 2 horas 
                    \item $<$ tiempo que se tardaría en recorrer la línea completa a 7,2 km/h
                \end{itemize}
                
                \item Velocidad:
                \begin{itemize}
                \tiny
                    \item $>=$ 0
                    \item $<$ 120 km/h
                \end{itemize}
            \end{itemize}
        
    \end{columns}
    
\end{frame} 


\section{Estimación y análisis de datos derivados}



\begin{frame}{Datos de campo y posición de los buses}
    \begin{columns}[T]
        \metroset{block=fill}
        \column{0.55\textwidth}
            \begin{exampleblock}{\scriptsize Datos de campo o sobre el terreno}
                \tiny Captura de las coordenadas de teléfono en trayecto desde ``Cuatro Caminos'' hasta ``ETSIT UPM'' (Línea F):
                \begin{itemize}
                    \item Registro de instante de apertura de puertas al llegar a cada parada.
                \end{itemize}
                Datos escasos debido al COVID-19.
            \end{exampleblock}
            
            \vspace{3em}
            
            \begin{block}{\scriptsize Estimación de la posición}
                \tiny 
                \begin{itemize}
                    \item Calculada a partir de $D^s_b$ y datos estáticos de la línea.
                    \item No empleada en algoritmos, solo aporta idea general de la ubicación de los buses.
                \end{itemize}
            \end{block}
            \vspace{1.5em}
            \tiny 
            
        \column{0.05\textwidth}    
        
        \column{0.48\textwidth}
            
            \vspace{0.65em}
            
            \includegraphics[width=0.95\textwidth]{images/trackeddata.png}
            
            \tiny Ruta seguida (morado) y paradas (verde) obtenidas con el teléfono.
            
            \vspace{1em}
            {\centering 
            \textbf{$\downarrow$ La distancia de la posición estimada a la real es de 150 m en media} 
            }
            \vspace{1em}
            
            \includegraphics[width=0.95\textwidth]{images/calculatedpaths.png}
            
            \tiny Ruta del bus obtenida con nuestro estimador de la posición para $T_m = 50$ s (rojo) y $T_m = 5$ s (azul).
            
    \end{columns}
\end{frame}


\begin{frame}{Tiempos de llegada: $T^a_{s,l,b}$}
    
    \begin{columns}[T]
        \metroset{block=fill}
        \tiny
        \column{0.35\textwidth}
            \begin{block}{\scriptsize Estimación de $T^a_{s,l,b}$}
                Tiempo de llegada del bus $b$ de la línea $l$ a la parada $s$.
                
                Calculado a partir de la primera instancia del bus que esté a menos de 45 s de la parada:
                \begin{itemize}
                    \item Error menor a 5 s.
                    \item Evitamos errores mayores (tiempo de parada elevado).
                \end{itemize}
            \end{block}
        \column{0.75\textwidth}
            \includegraphics[width=1\textwidth]{images/arrivaltimes.png}
    \end{columns}
    
    \vspace{1.5em}
    
    \begin{columns}[T]
        \column{0.4\textwidth}
            \centering \footnotesize
            \begin{table}
                \begin{tabular}{@{} lcc @{}}
                  \toprule
                  & $D^s_b<1km$ & $1km<D^s_b<2km$ \\
                  \midrule
                  $\bar{\epsilon}$ & -4,62 s & -12,48 s \\
                  $s_\epsilon$ & 51,57 s & 80,52 s \\
                  \bottomrule
                \end{tabular}
            \end{table}
            
            \tiny $\bar{\epsilon}$ y $s_\epsilon$ del error de las estimaciones de la API con respecto a nuestras estimaciones.
        
        \column{0.6\textwidth}
            \centering
            \includegraphics[width=0.85\textwidth]{images/arrivestimerror.png}
            
            \tiny Error respecto a los datos de campo para $T_m = 50s$ (rojo) y $T_m = 5s$ (azul).
    \end{columns}
\end{frame}


\begin{frame}{Tiempos entre paradas: $RT^{s_1,s_2}$}
    
    \centering Estimación del tiempo entre dos paradas consecutivas $s_1$ y $s_2$
    \begin{equation}
        \boxed{RT^{s_1,s_2} = T^{a}_{s_2,l,b} - T^{a}_{s_1,l,b}}\nonumber
    \end{equation}
    
    
    \vspace{3em}
    
    {\small $\bar{\epsilon}$ y $s_\epsilon$ de los tiempos entre paradas dados por la API}
    \begin{table}
        \begin{tabular}{@{} lr @{}}
          \toprule
          $\bar{\epsilon}$ & -8,16 s \\
          $s_\epsilon$ & 65,53 s \\
          \bottomrule
        \end{tabular}
    \end{table}
    
    
\end{frame}


\begin{frame}{Definición e importancia de los headways: $\hw{1}{2}$}
    \metroset{block=fill}
    \vspace{1em}
    \begin{block}{\small \textit{Headway}: intervalo de tiempo entre buses consecutivos $b_1$ y $b_2$}
        \scriptsize $\uparrow$ Regularidad de headways $\Rightarrow$ $\downarrow$ Tiempo de espera de los pasajeros en las paradas.
    \end{block}
    
    {\centering \includegraphics[width=1\textwidth]{images/headwayssimple.png}}
    
    \begin{alertblock}{\small Inestabilidad de headways}
        \scriptsize
        Si tenemos los buses $b_1$, $b_2$ y $b_3$:
        \begin{itemize}
            \item Retraso de $b_2$ $\Rightarrow$ $\uparrow \hw{1}{2}$ y $\downarrow \hw{2}{3}$. 
            \item Adelanto de $b_2$ $\Rightarrow$ $\downarrow \hw{1}{2}$ y $\uparrow \hw{2}{3}$.
        \end{itemize}
    \end{alertblock}
    
\end{frame}


\begin{frame}{Estimación de los headways: $\hw{1}{2}$}
    \metroset{block=fill} \vspace{0.5em}
    %{\tiny $T^s_b$ $\equiv$ Tiempo restante de bus $b$ para llegar a parada $s$}
    \begin{block}{\small Estimación de $\hw{1}{2}$}
        \footnotesize \centering
        ($RT^{s_1,s_2}$ y $T^s_b$) $\rightarrow$ $TTLS_{b}$ (tiempo restante para llegar a la última parada)
    \end{block}
    
    \begin{equation}
        \boxed{\hw{1}{2} = \overline{TTLS_{b_2}} - \overline{TTLS_{b_1}}}\nonumber
    \end{equation} 
    
    \centering \scriptsize
    \includegraphics[width=1\textwidth]{images/headways.png}
\end{frame} 


\section{Detección de anomalías en tiempo real sobre los headways}



\begin{frame}{Detección de anomalías en los headways}
    \metroset{block=fill}
    
    \begin{block}{\scriptsize Importancia}
        \scriptsize Mejora la eficiencia del transporte público.
        
        Ayuda a determinar causas y buscar soluciones.
    \end{block}
    
    \begin{block}{\scriptsize Problema de aprendizaje no supervisado}
        \scriptsize 
        Buscamos patrones anómalos en datos que no han sido previamente etiquetados.
    \end{block}
    \centering
    \includegraphics[width=0.85\textwidth]{images/unsupervised.png}
    %\includegraphics[width=0.85\textwidth]{images/unsupervised2.jpg}
\end{frame}

\def\hwvector{
    \begin{bmatrix}
       \hw{1}{2} & \hw{2}{3} & \cdots & \hw{m-1}{m}
    \end{bmatrix}
}


\begin{frame}{Enfoques al problema}
    
    \Large Construyen el concepto de \\ \textbf{Series k-Dimensionales de Headways}:
    
    \begin{columns}[T]
        \metroset{block=fill}
        \column{0.5\textwidth}
            
            \begin{block}{$1.^{\rm{er}}$ enfoque}
                \scriptsize $b_1$ $\equiv$ bus más cercano al final 
                \\[1em]
                Adelantamientos $\Rightarrow$ desaparición de series, así como la aparición de nuevas series.
            \end{block}
            
        \column{0.5\textwidth}
            \begin{exampleblock}{$2.^{\rm{o}}$ enfoque}
                \scriptsize $b_1$ $\equiv$ bus que entró en primer lugar
                \\[1em]
                Adelantamientos $\Rightarrow$ headways negativos.
            \end{exampleblock}
        
    \end{columns}
    
    \vspace{1em}
    
    \normalsize 
    \textbf{Conceptos comunes:}
    \begin{itemize}
        \item Vector de headways.
        \item Ventana deslizante sobre el vector de headways.
        \item Series temporales de headways.
    \end{itemize}
    
\end{frame}


\begin{frame}{Vector de headways}
    \metroset{block=fill}
    \centering
    
    \begin{alertblock}{\small Alta correlación entre headways consecutivos}
        \scriptsize $\hw{1}{2}$ y $\hw{2}{3}$ tienen $b_2$ en común.
    \end{alertblock}
    
    \scriptsize \textbf{$\downarrow$ Cuantos más headways analicemos conjuntamente, más información tenemos para considerar si son o no anómalos}
    
    \begin{columns}[T]
        \column{0.33\textwidth}
            
            \begin{exampleblock}{\small Vector de headways}
                \scriptsize 
                \textbf{-} 1 por cada una de las 2 direcciones $d$ de la línea $l$.
                
                \textbf{-} Se actualiza cada instante de muestreo $t_i$.
            \end{exampleblock}
            \vspace{2em}
            \footnotesize $m$ $\equiv$ número de buses circulando por la dirección de la línea
        \column{0.75\textwidth}
            \small
            \begin{gather}
                \boxed{H_{l,d,m}(t_i) = \hwvector} \nonumber \\
                \forall m \in \mathbb{N} , m>1 \nonumber
            \end{gather}
            
            
            \centering 
            \includegraphics[width=1\textwidth]{images/headwaysvector.png}
            \centering \scriptsize Ejemplo vector de headways.
            
    \end{columns}
    
\end{frame}

\def\hwslice{
    \begin{bmatrix}
        \hw{i}{i+1} & \hw{i+1}{i+2} & \cdots & \hw{k+i-1}{k+i}
    \end{bmatrix}
}

\begin{frame}{Ventana deslizante sobre el vector de headways}
    \metroset{block=fill}
    \centering
    
    \begin{alertblock}{\small $m$ varía con el tiempo}
        \scriptsize Frecuentemente entre 2 valores si acotamos lo suficiente el intervalo de análisis.
    \end{alertblock}
    
    \scriptsize \textbf{$\downarrow$ Necesitamos ser capaces de analizar vectores con $m$ variable}
    
    \begin{columns}[T]
        \column{0.35\textwidth}
            \begin{exampleblock}{\small Ventana deslizante sobre el vector de headways}
                \scriptsize Divide el vector de headways en rodajas de tamaño $k$:
                
                \textbf{-} ($m-k$) rodajas por vector.
                
                \textbf{-} Aplicable si $m \geq k+1$.
            \end{exampleblock}
            \vspace{2em} 
            \centering \small
            k $\equiv$ D \\
            $\uparrow$ k $\Rightarrow$ $\downarrow$ $N^{sw}_{k}$
            
        \column{0.72\textwidth}
            \footnotesize 
            \begin{gather}
                \boxed{H^{slice}_{l,d,m,k,i} = \hwslice} \nonumber \\ 
                1 \leq k \leq m-1 ; \; \; 1 \leq i \leq m-k \nonumber
            \end{gather}
            
            \centering 
            \includegraphics[width=1\textwidth]{images/headwayssw.png}
            \scriptsize Ejemplo ventana deslizante k=2.
            
        
    \end{columns}
    
\end{frame}


\begin{frame}{Línea 132 en días laborables de 10am a 11am: Ejemplo 1}
    %\metroset{block=fill}
    \centering Aplicación de las ventanas deslizantes
     
    \begin{columns}[T]
        \tiny
        \column{0.37\textwidth}
            \centering
            \includegraphics[width=1\textwidth]{images/buses132-1011.png}
            
            Número de vectores de headways para cada tamaño $(m-1)$: $N_{(m-1)}$.
            
        \column{0.26\textwidth}
            \centering
            \begin{gather}
                \boxed{N^{sw}_{k} = \sum_{m=k+1}^{m_{\max}+1} (m-k)N_{(m-1)}} \nonumber \\
                1 \leq k \leq m_{\max}-1 \nonumber
            \end{gather}
        \column{0.37\textwidth}
            \centering
            \includegraphics[width=1\textwidth]{images/slices132-1011.png}
            
            Número de rodajas de headways para cada tamaño de ventana deslizante $k$: $N^{sw}_{k}$.
            
    \end{columns}
    
    \vspace{1.5em}
    
    \begin{columns}[T]

        \column{0.45\textwidth}
            \centering Distribución de los datos de cada ventana deslizante
            
            \includegraphics[width=1\textwidth]{images/slices-1-density.png}
            
            \tiny Distribución de $N^{sw}_{1}$.
            
        \column{0.45\textwidth}
            \centering
            \includegraphics[width=0.75\textwidth]{images/slices-2-density.png}
            
            \tiny Distribución de $N^{sw}_{2}$.
            
    \end{columns}

\end{frame}

\def\busgroup{
    \begin{bmatrix}
       b_1 & b_2 & \cdots & b_k & b_{(k+1)}
    \end{bmatrix}
}

\def\hwgroup{
    \begin{bmatrix}
       \hw{1}{2}(t_i) & \hw{2}{3}(t_i) & \cdots & \hw{k}{(k+1)}(t_i)
    \end{bmatrix}
}

\begin{frame}{Series temporales de headways}
    \metroset{block=fill}
    \begin{exampleblock}{\small Serie temporal de headways}
        \scriptsize Una serie temporal de headways está formada por todas las rodajas $H^{slice}_{l,d,m,k,i}$ que corresponden al grupo de buses consecutivos $g_k = \busgroup$:
        
        \begin{center}
            $\uparrow$ dimensión $\Rightarrow$ $\downarrow$ duración de la serie
        \end{center}
    \end{exampleblock}
    
    \small
    \begin{gather}
        \boxed{H_{g_k}(t_i) = \hwgroup} \nonumber
    \end{gather}
    
    \vspace{3em}
    \normalsize
    \textbf{Series k-Dimensionales de Headways:} conjunto de todas las series temporales de tamaño $k$.
    
\end{frame}


\begin{frame}{Línea 132 en días laborables de 10am a 11am: Ejemplo 2}
    %\metroset{block=fill}
    \centering
    \begin{columns}[T]

        \column{0.59\linewidth}
            \centering \vspace{2.5em}
            \scriptsize Muestra de las Series 1-D obtenidas.
            \includegraphics[width=1.1\textwidth]{images/slices-1-series.png}
            
            
        \column{0.45\linewidth}
            \centering
            \scriptsize Muestra de las Series 2-D obtenidas {\tiny (variable temporal proyectada en el espacio bidimensional de variables)}.
            \includegraphics[width=1\textwidth]{images/slices-2-series.png}
            
            
    \end{columns}

    Análogo para mayores dimensiones. En este ejemplo podemos construir un modelo de detección de anomalías como máximo de dimensión 7, ya que $N^{sw}_{6} = 147$,  $N^{sw}_{7} = 6$ y $N^{sw}_{8} = 0$.
\end{frame}


\begin{frame}{Modelo de detección de anomalías (i)}
    \metroset{block=fill}
    \begin{columns}[T]
        
        \column{0.5\linewidth}
            \centering \small
            \textbf{Asumimos que las series N-Dimensionales de headways siguen aproximadamente una distribución normal N-dimensional.}
            
            $\downarrow$
            
            \begin{block}{\footnotesize Distancia de Mahalanobis ($D_M$)} 
                \scriptsize Desviaciones estándar que separan rodaja $\vec{x}$ del conjunto de rodajas D (con media $\vec{\mu}$ y matriz de covarianza $S$).
            \end{block}
            
            \begin{equation}
                \boxed{D_M(\vec{x}) = \sqrt{(\vec{x}-\vec{\mu})^T S^{-1} (\vec{x}-\vec{\mu})}} \nonumber
            \end{equation}

        \column{0.55\linewidth}
            \centering \footnotesize
            \begin{block}{\scriptsize Región de confianza}
                \scriptsize Podemos calcular un valor umbral de $D_M$ que contenga un porcentaje deseado de rodajas.
            \end{block}
            
            \includegraphics[width=1\textwidth]{images/conf-ellipses.png}
            \scriptsize Elipses de confianza (k=2).
            
    \end{columns}
    
\end{frame}


\begin{frame}{Modelo de detección de anomalías (ii)}
    \metroset{block=fill}
    \begin{columns}[T]
    
        \column{0.55\linewidth}
            \begin{block}{\small Series temporales de $D_M$}
                \scriptsize Calculamos $D_M$ para cada muestra de las series temporales de headways.
            \end{block}
            
            \begin{block}{\scriptsize Hiperparámetros}
                \scriptsize 
                \begin{itemize}
                    \item \textbf{Confianza ($conf$):} umbral $D_{M,th}$ a partir del cuál una subserie es potencialmente anómala.
                    \item \textbf{Umbral de duración ($dur\textunderscore th$):} mínimo número de muestras consecutivas por encima de $D_{M,th}$ para detectar la subserie como anómala.
                \end{itemize}
                
            \end{block}
            
            
        \column{0.55\linewidth}
            \centering 
            \includegraphics[width=1\textwidth]{images/detected-2d-anoms.png}
            
            \scriptsize Subseries anómalas detectadas con $conf=95\%$ y $dur\textunderscore th= 1$ en las series 2-D del ejemplo 2.
            
            
    \end{columns}
    
\end{frame}



\begin{frame}{Línea 132 en días laborables de 10am a 11am: Ejemplo 3}
    \metroset{block=fill}
    \begin{columns}[T]
        \column{0.5\textwidth}
            \scriptsize
            \begin{block}{Parámetros de rendimiento:}
                \begin{itemize}
                    \item $N_{anom}$: número de subseries anómalas detectadas en las series de headways.
                    \item $\overline{dur}$: duración media de las subseries anómalas detectadas.
                    \item $p_{anom}$: porcentaje de las rodajas que forman todas las subseries que han sido detectadas como anomalías.
                \end{itemize}
            \end{block}
            \vspace{2em}
            
        \column{0.5\textwidth}
            \centering \footnotesize \vspace{1em}
            Resultados al aplicar nuestro detector de anomalías con \textbf{conf=90\%}:
            \vspace{1em}
            \begin{table}
                \begin{tabular}{@{} lccr @{}}
                  \toprule
                  \textbf{dur\textunderscore th} & $N_{anom}$ & $\overline{dur}$ & $p_{anom}$ \\
                  \midrule
                  \textbf{1} & 95 & 6,1 & 9,4\% \\
                  \textbf{2} & 66 & 8,3 & 9,0\% \\
                  \bottomrule
                \end{tabular}
            \end{table}
            Series 1-D ($N^{sw}_1 = 6138$).
            
            \vspace{1em}
            \begin{table}
                \begin{tabular}{@{} lccr @{}}
                  \toprule
                  \textbf{dur\textunderscore th} & $N_{anom}$ & $\overline{dur}$ & $p_{anom}$ \\
                  \midrule
                  \textbf{1} & 79 & 6,0 & 9,8\% \\
                  \textbf{2} & 54 & 8,3 & 9,3\% \\
                  \bottomrule
                \end{tabular}
            \end{table}
            Series 2-D ($N^{sw}_2 = 4797$).
            
    \end{columns}
    
\end{frame} 


\section{Influencia del COVID-19}



\begin{frame}{Efecto en los tiempos entre paradas y los headways}
    \metroset{block=fill}
    
    \begin{columns}[T]
        \column{0.4\textwidth}
            \scriptsize
            \begin{block}{\scriptsize Separación de los datos}
                \textbf{- Datos recogidos antes del COVID-19 ($X_{b}$):} del 26 de Febrero al 15 de Marzo.
                
                \textbf{- Datos recogidos durante el COVID-19 ($X_{w}$):} del 15 de Marzo al 10 de Abril.
            \end{block}
            
            \scriptsize
            \begin{block}{\scriptsize Test de hipótesis de Aspin-Welch:}
                Comprobamos si la distribución de $X_{b}$ se asemeja a la de $X_{w}$:
                \begin{align*}
                    H_{0}:\; \mu_b = \mu_w \\
                    H_{1}:\; \mu_b \neq \mu_w
                \end{align*}
            \end{block}
            
        \column{0.6\textwidth}
            \centering \scriptsize \vspace{1em}
            {\footnotesize Test sobre los tiempos entre paradas}
            \begin{table}
                \begin{tabular}{@{} lcccr @{}}
                  \toprule
                  $R.V.$ & $n$ & $\overline{x}$ & $s$ & $p-valor$ \\
                  \midrule
                  $X_{b}$ & 189 & 139,2 s & 70,0 s &  \\
                  $X_{w}$ & 228 & 64,1 s & 34,1 s & \multirow{-2}{*}{$\scinot{6.3}{-30}$} \\
                  \bottomrule
                \end{tabular}
            \end{table}
            
            \vspace{1em}
            
            {\footnotesize Test sobre los headways}
            \begin{table}
                \begin{tabular}{@{} lcccr @{}}
                  \toprule
                  $R.V.$ & $n$ & $\overline{x}$ & $s$ & $p-valor$ \\
                  \midrule
                  $X_{b}$ & 19101 & 606,4 s & 356,8 s &  \\
                  $X_{w}$ & 18680 & 726,4 s & 329,5 s & \multirow{-2}{*}{$\scinot{4.3}{-249}$} \\
                  \bottomrule
                \end{tabular}
            \end{table}
            
            \vspace{1em}
            
            {\footnotesize 
            $\downarrow$ desviaciones estándar
            \\[1em]
            $\downarrow$ tiempos entre paradas
            \\[1em]
            $\uparrow$ headways
            }
    \end{columns}
    
\end{frame}


\begin{frame}{Efecto en los headways según nuestro detector de anomalías}
    \metroset{block=fill}
    \begin{columns}[T]
        \column{0.5\textwidth}
            \centering \footnotesize \vspace{1em}
            Resultados al aplicar nuestro detector de anomalías con  \textbf{conf=90\%}:
            \vspace{1em}
            \begin{table}
                \begin{tabular}{@{} lccr @{}}
                  \toprule
                  \textbf{dur\textunderscore th} & $N_{anom}$ & $\overline{dur}$ & $p_{anom}$ \\
                  \midrule
                  \textbf{1} & 92 & 4,4 & 3,0\% \\
                  \textbf{2} & 64 & 6,0 & 2,7\% \\
                  \bottomrule
                \end{tabular}
            \end{table}
            Series 1-D ($N^{sw}_1 = 13937$).
            
            \vspace{1em}
            \begin{table}
                \begin{tabular}{@{} lccr @{}}
                  \toprule
                  \textbf{dur\textunderscore th} & $N_{anom}$ & $\overline{dur}$ & $p_{anom}$ \\
                  \midrule
                  \textbf{1} & 61 & 3,1 & 2,7\% \\
                  \textbf{2} & 41 & 4,1 & 2,4\% \\
                  \bottomrule
                \end{tabular}
            \end{table}
            Series 2-D ($N^{sw}_2 = 7093$).
            
        \column{0.5\textwidth}
            \footnotesize Distribución de $N^{sw}_{2}$ antes (azul) y durante (rojo) el COVID-19.
            \includegraphics[width=1\textwidth]{images/anom2dcovid.png}
            No se ve afectado por cambios que provocan un comportamiento más regular de los headways.
            
            
    \end{columns}
\end{frame} 


\section{Dashboard}



\begin{frame}{Elementos e interacción}
    %\metroset{block=fill}
    Accesible desde el enlace: \href{http://euler2.mat.upm.es/realtime/132}{euler2.mat.upm.es/realtime/132}
    
    {\centering
    \includegraphics[width=1\textwidth]{images/realtime-dashboard.jpg}
    }
    \scriptsize
    \begin{columns}[T]
        \column{0.33\textwidth}
            \textbf{1. Selector de hiperparámetros.}\\
            
            \textbf{2. Tiempo restante de los buses para llegar al destino.}
        \column{0.33\textwidth}
            \textbf{3. Posición de los buses.}\\
            
            \textbf{4. Series 1-D de Headways.}\\
            
            \textbf{5. Series 2-D de Headways.}
        \column{0.33\textwidth}
            \textbf{6. Series de distancias de Mahalanobis.}\\
            
            \textbf{7. Anomalías detectadas.}
            
    \end{columns}
    
\end{frame} 


\section{Conclusiones y futuras líneas de trabajo}



\begin{frame}{Conclusiones}
        \begin{itemize}
            \item Adquisición de conocimientos valiosos:
            \begin{itemize}
                \item Análisis y limpieza de datos.
                \item Visualización.
                \item Aprendizaje automático no supervisado.
                \item Funcionamiento del transporte público (headways).
            \vspace{1em}
            \end{itemize}
            \item Las anomalías detectadas pueden ayudar a mejorar la eficiencia del servicio prestado por los buses.
        \end{itemize}
\end{frame}


\begin{frame}{Futuras líneas de trabajo}
        \begin{itemize}
            \item Caracterizar el papel de la dimensión de las series de headways en la detección de anomalías.
            \vspace{1em}
            \item Extraer otras características de las series temporales  (variabilidad, correlación con otras series, etc.) para mejorar el esquema de detección de anomalías.
            \vspace{1em}
            \item Implementar un clasificador de las anomalías detectadas, que indique su causa.
        \end{itemize}
\end{frame}


\appendix



\begin{frame}{Agradecimientos}
    \centering{\huge{¡Muchas gracias!}}
    
    \vskip 1.5cm
        
    \includegraphics[height=1.5cm]{images/etsitupm.png}
    \quad \quad
    \includegraphics[height=1.5cm]{images/cabify.png}
\end{frame}

{\setbeamercolor{palette primary}{fg=black, bg=cyan}
\begin{frame}[standout]
  \centering{\huge{¿Alguna pregunta?}}
\end{frame}
}

\begin{frame}[allowframebreaks]{Backup}

    \begin{block}{Estimación de la posición}
        \small
        \begin{enumerate}
            \item Seleccionamos parada con mínimo $D^s_b$ y obtenemos su distancia al inicio de la línea ($D^s$).
            \item Calculamos distancia del bus al inicio de la línea: $D_b = D^s - D^s_b$.
            \item Buscamos el punto $P$ de la línea cuya distancia al comienzo sea más cercana a $D_b$. 
            \item Usamos las coordenadas de $P$ como estimación de la ubicación del bus.
        \end{enumerate}
    \end{block}
    
    \centering \includegraphics[width=1\textwidth]{images/headwaysequality.png}
    
    \small Ejemplo de la inestabilidad de los headways debido al número variable de pasajeros en cada parada \cite{gershenson2009}.
    
    \centering \includegraphics[width=1\textwidth]{images/hwovertime.png}
    
    \small Desaparición y aparición de series en adelantamientos. 
    
    \begin{table}
        \begin{tabular}{@{} lr @{}}
          \toprule
          & Tiempo \\
          \midrule
          Limpieza & 1,21 h \\
          $T^a_{s,l,b}$ & 4,29 h \\
          $RT^{s_1,s_2}$ & 0,70 h\\
          $\hw{1}{2}$  & 7,88 h \\ 
          $S^{-1}$ y $\vec{\mu}$  & 0,72 h \\
          \bottomrule
        \end{tabular}
    \end{table}
    
    \small Tiempos de procesamiento partiendo de 7,05 GB de datos.
    
    Usamos los 8 núcleos disponibles en la máquina con técnicas de procesamiento paralelo.
    
\end{frame}

\begin{frame}[allowframebreaks]{References}

  \bibliography{references}
  \bibliographystyle{abbrv}

\end{frame}



\end{document}
