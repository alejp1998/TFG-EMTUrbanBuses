

\begin{frame}{Detección de anomalías en los headways}
    \metroset{block=fill}
    
    \begin{block}{\scriptsize Importancia}
        \scriptsize Mejora la eficiencia del transporte público.
        
        Ayuda a determinar causas y buscar soluciones.
    \end{block}
    
    \begin{block}{\scriptsize Problema de aprendizaje no supervisado}
        \scriptsize 
        Buscamos patrones anómalos en datos que no han sido previamente etiquetados.
    \end{block}
    \centering
    \includegraphics[width=0.85\textwidth]{images/unsupervised.png}
    %\includegraphics[width=0.85\textwidth]{images/unsupervised2.jpg}
\end{frame}

\def\hwvector{
    \begin{bmatrix}
       \hw{1}{2} & \hw{2}{3} & \cdots & \hw{m-1}{m}
    \end{bmatrix}
}


\begin{frame}{Enfoques al problema}
    
    \Large Construyen el concepto de \\ \textbf{Series k-Dimensionales de Headways}:
    
    \begin{columns}[T]
        \metroset{block=fill}
        \column{0.5\textwidth}
            
            \begin{block}{$1.^{\rm{er}}$ enfoque}
                \scriptsize $b_1$ $\equiv$ bus más cercano al final 
                \\[1em]
                Adelantamientos $\Rightarrow$ desaparición de series, así como la aparición de nuevas series.
            \end{block}
            
        \column{0.5\textwidth}
            \begin{exampleblock}{$2.^{\rm{o}}$ enfoque}
                \scriptsize $b_1$ $\equiv$ bus que entró en primer lugar
                \\[1em]
                Adelantamientos $\Rightarrow$ headways negativos.
            \end{exampleblock}
        
    \end{columns}
    
    \vspace{1em}
    
    \normalsize 
    \textbf{Conceptos comunes:}
    \begin{itemize}
        \item Vector de headways.
        \item Ventana deslizante sobre el vector de headways.
        \item Series temporales de headways.
    \end{itemize}
    
\end{frame}


\begin{frame}{Vector de headways}
    \metroset{block=fill}
    \centering
    
    \begin{alertblock}{\small Alta correlación entre headways consecutivos}
        \scriptsize $\hw{1}{2}$ y $\hw{2}{3}$ tienen $b_2$ en común.
    \end{alertblock}
    
    \scriptsize \textbf{$\downarrow$ Cuantos más headways analicemos conjuntamente, más información tenemos para considerar si son o no anómalos}
    
    \begin{columns}[T]
        \column{0.33\textwidth}
            
            \begin{exampleblock}{\small Vector de headways}
                \scriptsize 
                \textbf{-} 1 por cada una de las 2 direcciones $d$ de la línea $l$.
                
                \textbf{-} Se actualiza cada instante de muestreo $t_i$.
            \end{exampleblock}
            \vspace{2em}
            \footnotesize $m$ $\equiv$ número de buses circulando por la dirección de la línea
        \column{0.75\textwidth}
            \small
            \begin{gather}
                \boxed{H_{l,d,m}(t_i) = \hwvector} \nonumber \\
                \forall m \in \mathbb{N} , m>1 \nonumber
            \end{gather}
            
            
            \centering 
            \includegraphics[width=1\textwidth]{images/headwaysvector.png}
            \centering \scriptsize Ejemplo vector de headways.
            
    \end{columns}
    
\end{frame}

\def\hwslice{
    \begin{bmatrix}
        \hw{i}{i+1} & \hw{i+1}{i+2} & \cdots & \hw{k+i-1}{k+i}
    \end{bmatrix}
}

\begin{frame}{Ventana deslizante sobre el vector de headways}
    \metroset{block=fill}
    \centering
    
    \begin{alertblock}{\small $m$ varía con el tiempo}
        \scriptsize Frecuentemente entre 2 valores si acotamos lo suficiente el intervalo de análisis.
    \end{alertblock}
    
    \scriptsize \textbf{$\downarrow$ Necesitamos ser capaces de analizar vectores con $m$ variable}
    
    \begin{columns}[T]
        \column{0.35\textwidth}
            \begin{exampleblock}{\small Ventana deslizante sobre el vector de headways}
                \scriptsize Divide el vector de headways en rodajas de tamaño $k$:
                
                \textbf{-} ($m-k$) rodajas por vector.
                
                \textbf{-} Aplicable si $m \geq k+1$.
            \end{exampleblock}
            \vspace{2em} 
            \centering \small
            k $\equiv$ D \\
            $\uparrow$ k $\Rightarrow$ $\downarrow$ $N^{sw}_{k}$
            
        \column{0.72\textwidth}
            \footnotesize 
            \begin{gather}
                \boxed{H^{slice}_{l,d,m,k,i} = \hwslice} \nonumber \\ 
                1 \leq k \leq m-1 ; \; \; 1 \leq i \leq m-k \nonumber
            \end{gather}
            
            \centering 
            \includegraphics[width=1\textwidth]{images/headwayssw.png}
            \scriptsize Ejemplo ventana deslizante k=2.
            
        
    \end{columns}
    
\end{frame}


\begin{frame}{Línea 132 en días laborables de 10am a 11am: Ejemplo 1}
    %\metroset{block=fill}
    \centering Aplicación de las ventanas deslizantes
     
    \begin{columns}[T]
        \tiny
        \column{0.37\textwidth}
            \centering
            \includegraphics[width=1\textwidth]{images/buses132-1011.png}
            
            Número de vectores de headways para cada tamaño $(m-1)$: $N_{(m-1)}$.
            
        \column{0.26\textwidth}
            \centering
            \begin{gather}
                \boxed{N^{sw}_{k} = \sum_{m=k+1}^{m_{\max}+1} (m-k)N_{(m-1)}} \nonumber \\
                1 \leq k \leq m_{\max}-1 \nonumber
            \end{gather}
        \column{0.37\textwidth}
            \centering
            \includegraphics[width=1\textwidth]{images/slices132-1011.png}
            
            Número de rodajas de headways para cada tamaño de ventana deslizante $k$: $N^{sw}_{k}$.
            
    \end{columns}
    
    \vspace{1.5em}
    
    \begin{columns}[T]

        \column{0.45\textwidth}
            \centering Distribución de los datos de cada ventana deslizante
            
            \includegraphics[width=1\textwidth]{images/slices-1-density.png}
            
            \tiny Distribución de $N^{sw}_{1}$.
            
        \column{0.45\textwidth}
            \centering
            \includegraphics[width=0.75\textwidth]{images/slices-2-density.png}
            
            \tiny Distribución de $N^{sw}_{2}$.
            
    \end{columns}

\end{frame}

\def\busgroup{
    \begin{bmatrix}
       b_1 & b_2 & \cdots & b_k & b_{(k+1)}
    \end{bmatrix}
}

\def\hwgroup{
    \begin{bmatrix}
       \hw{1}{2}(t_i) & \hw{2}{3}(t_i) & \cdots & \hw{k}{(k+1)}(t_i)
    \end{bmatrix}
}

\begin{frame}{Series temporales de headways}
    \metroset{block=fill}
    \begin{exampleblock}{\small Serie temporal de headways}
        \scriptsize Una serie temporal de headways está formada por todas las rodajas $H^{slice}_{l,d,m,k,i}$ que corresponden al grupo de buses consecutivos $g_k = \busgroup$:
        
        \begin{center}
            $\uparrow$ dimensión $\Rightarrow$ $\downarrow$ duración de la serie
        \end{center}
    \end{exampleblock}
    
    \small
    \begin{gather}
        \boxed{H_{g_k}(t_i) = \hwgroup} \nonumber
    \end{gather}
    
    \vspace{3em}
    \normalsize
    \textbf{Series k-Dimensionales de Headways:} conjunto de todas las series temporales de tamaño $k$.
    
\end{frame}


\begin{frame}{Línea 132 en días laborables de 10am a 11am: Ejemplo 2}
    %\metroset{block=fill}
    \centering
    \begin{columns}[T]

        \column{0.59\linewidth}
            \centering \vspace{2.5em}
            \scriptsize Muestra de las Series 1-D obtenidas.
            \includegraphics[width=1.1\textwidth]{images/slices-1-series.png}
            
            
        \column{0.45\linewidth}
            \centering
            \scriptsize Muestra de las Series 2-D obtenidas {\tiny (variable temporal proyectada en el espacio bidimensional de variables)}.
            \includegraphics[width=1\textwidth]{images/slices-2-series.png}
            
            
    \end{columns}

    Análogo para mayores dimensiones. En este ejemplo podemos construir un modelo de detección de anomalías como máximo de dimensión 7, ya que $N^{sw}_{6} = 147$,  $N^{sw}_{7} = 6$ y $N^{sw}_{8} = 0$.
\end{frame}


\begin{frame}{Modelo de detección de anomalías (i)}
    \metroset{block=fill}
    \begin{columns}[T]
        
        \column{0.5\linewidth}
            \centering \small
            \textbf{Asumimos que las series N-Dimensionales de headways siguen aproximadamente una distribución normal N-dimensional.}
            
            $\downarrow$
            
            \begin{block}{\footnotesize Distancia de Mahalanobis ($D_M$)} 
                \scriptsize Desviaciones estándar que separan rodaja $\vec{x}$ del conjunto de rodajas D (con media $\vec{\mu}$ y matriz de covarianza $S$).
            \end{block}
            
            \begin{equation}
                \boxed{D_M(\vec{x}) = \sqrt{(\vec{x}-\vec{\mu})^T S^{-1} (\vec{x}-\vec{\mu})}} \nonumber
            \end{equation}

        \column{0.55\linewidth}
            \centering \footnotesize
            \begin{block}{\scriptsize Región de confianza}
                \scriptsize Podemos calcular un valor umbral de $D_M$ que contenga un porcentaje deseado de rodajas.
            \end{block}
            
            \includegraphics[width=1\textwidth]{images/conf-ellipses.png}
            \scriptsize Elipses de confianza (k=2).
            
    \end{columns}
    
\end{frame}


\begin{frame}{Modelo de detección de anomalías (ii)}
    \metroset{block=fill}
    \begin{columns}[T]
    
        \column{0.55\linewidth}
            \begin{block}{\small Series temporales de $D_M$}
                \scriptsize Calculamos $D_M$ para cada muestra de las series temporales de headways.
            \end{block}
            
            \begin{block}{\scriptsize Hiperparámetros}
                \scriptsize 
                \begin{itemize}
                    \item \textbf{Confianza ($conf$):} umbral $D_{M,th}$ a partir del cuál una subserie es potencialmente anómala.
                    \item \textbf{Umbral de duración ($dur\textunderscore th$):} mínimo número de muestras consecutivas por encima de $D_{M,th}$ para detectar la subserie como anómala.
                \end{itemize}
                
            \end{block}
            
            
        \column{0.55\linewidth}
            \centering 
            \includegraphics[width=1\textwidth]{images/detected-2d-anoms.png}
            
            \scriptsize Subseries anómalas detectadas con $conf=95\%$ y $dur\textunderscore th= 1$ en las series 2-D del ejemplo 2.
            
            
    \end{columns}
    
\end{frame}



\begin{frame}{Línea 132 en días laborables de 10am a 11am: Ejemplo 3}
    \metroset{block=fill}
    \begin{columns}[T]
        \column{0.5\textwidth}
            \scriptsize
            \begin{block}{Parámetros de rendimiento:}
                \begin{itemize}
                    \item $N_{anom}$: número de subseries anómalas detectadas en las series de headways.
                    \item $\overline{dur}$: duración media de las subseries anómalas detectadas.
                    \item $p_{anom}$: porcentaje de las rodajas que forman todas las subseries que han sido detectadas como anomalías.
                \end{itemize}
            \end{block}
            \vspace{2em}
            
        \column{0.5\textwidth}
            \centering \footnotesize \vspace{1em}
            Resultados al aplicar nuestro detector de anomalías con \textbf{conf=90\%}:
            \vspace{1em}
            \begin{table}
                \begin{tabular}{@{} lccr @{}}
                  \toprule
                  \textbf{dur\textunderscore th} & $N_{anom}$ & $\overline{dur}$ & $p_{anom}$ \\
                  \midrule
                  \textbf{1} & 95 & 6,1 & 9,4\% \\
                  \textbf{2} & 66 & 8,3 & 9,0\% \\
                  \bottomrule
                \end{tabular}
            \end{table}
            Series 1-D ($N^{sw}_1 = 6138$).
            
            \vspace{1em}
            \begin{table}
                \begin{tabular}{@{} lccr @{}}
                  \toprule
                  \textbf{dur\textunderscore th} & $N_{anom}$ & $\overline{dur}$ & $p_{anom}$ \\
                  \midrule
                  \textbf{1} & 79 & 6,0 & 9,8\% \\
                  \textbf{2} & 54 & 8,3 & 9,3\% \\
                  \bottomrule
                \end{tabular}
            \end{table}
            Series 2-D ($N^{sw}_2 = 4797$).
            
    \end{columns}
    
\end{frame}