

\begin{frame}{Conclusiones}
        \begin{itemize}
            \item Adquisición de conocimientos valiosos:
            \begin{itemize}
                \item Análisis y limpieza de datos.
                \item Visualización.
                \item Aprendizaje automático no supervisado.
                \item Funcionamiento del transporte público (headways).
            \vspace{1em}
            \end{itemize}
            \item Las anomalías detectadas pueden ayudar a mejorar la eficiencia del servicio prestado por los buses.
        \end{itemize}
\end{frame}


\begin{frame}{Futuras líneas de trabajo}
        \begin{itemize}
            \item Caracterizar el papel de la dimensión de las series de headways en la detección de anomalías.
            \vspace{1em}
            \item Extraer otras características de las series temporales  (variabilidad, correlación con otras series, etc.) para mejorar el esquema de detección de anomalías.
            \vspace{1em}
            \item Implementar un clasificador de las anomalías detectadas, que indique su causa.
        \end{itemize}
\end{frame}