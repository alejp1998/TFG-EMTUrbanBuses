

\begin{frame}{Agradecimientos}
    \centering{\huge{¡Muchas gracias!}}
    
    \vskip 1.5cm
        
    \includegraphics[height=1.5cm]{images/etsitupm.png}
    \quad \quad
    \includegraphics[height=1.5cm]{images/cabify.png}
\end{frame}

{\setbeamercolor{palette primary}{fg=black, bg=cyan}
\begin{frame}[standout]
  \centering{\huge{¿Alguna pregunta?}}
\end{frame}
}

\begin{frame}[allowframebreaks]{Backup}

    \begin{block}{Estimación de la posición}
        \small
        \begin{enumerate}
            \item Seleccionamos parada con mínimo $D^s_b$ y obtenemos su distancia al inicio de la línea ($D^s$).
            \item Calculamos distancia del bus al inicio de la línea: $D_b = D^s - D^s_b$.
            \item Buscamos el punto $P$ de la línea cuya distancia al comienzo sea más cercana a $D_b$. 
            \item Usamos las coordenadas de $P$ como estimación de la ubicación del bus.
        \end{enumerate}
    \end{block}
    
    \centering \includegraphics[width=1\textwidth]{images/headwaysequality.png}
    
    \small Ejemplo de la inestabilidad de los headways debido al número variable de pasajeros en cada parada \cite{gershenson2009}.
    
    \centering \includegraphics[width=1\textwidth]{images/hwovertime.png}
    
    \small Desaparición y aparición de series en adelantamientos. 
    
    \begin{table}
        \begin{tabular}{@{} lr @{}}
          \toprule
          & Tiempo \\
          \midrule
          Limpieza & 1,21 h \\
          $T^a_{s,l,b}$ & 4,29 h \\
          $RT^{s_1,s_2}$ & 0,70 h\\
          $\hw{1}{2}$  & 7,88 h \\ 
          $S^{-1}$ y $\vec{\mu}$  & 0,72 h \\
          \bottomrule
        \end{tabular}
    \end{table}
    
    \small Tiempos de procesamiento partiendo de 7,05 GB de datos.
    
    Usamos los 8 núcleos disponibles en la máquina con técnicas de procesamiento paralelo.
    
\end{frame}

\begin{frame}[allowframebreaks]{References}

  \bibliography{references}
  \bibliographystyle{abbrv}

\end{frame}

