

\chapter{Abstract}

Public transport is an indispensable element today, helping to improve and make accessible to everyone urban mobility, as well as reducing the pollution inside big cities.\\

For this reason, in the present Final Degree Project we have performed an analysis of the data corresponding to the arrival times of the urban buses of the EMT, with the aim of improving their performance and reliability.\\

To achieve this, we have developed a series of algorithms to estimate derived data from the original data, such as the position, arrival times, running time between stops and headways of the buses, and we have characterized and analyzed its behaviour using statistical inference tools.\\

Making use of this derived data in conjunction with statistical techniques, we have designed and tested an unsupervised real-time anomaly detection model on the headways (time intervals between consecutive buses). The anomalies detected by this model constitute a valuable feedback of the quality of the service given by the buses, which can then be employed to make timely decisions to enhance this service.\\

Finally, we have developed a web visualization tool, in order to monitor the real-time anomaly detection framework and to make it easier to understand the developed tools.\\ \\

\textbf{Keywords:} public transport, urban mobility, data analysis, data treatment, data science, buses, statistics, headways, unsupervised machine learning, statistical inference, anomalies, visualization. 


\chapter{Resumen}

El transporte público es hoy en día un elemento indispensable que contribuye a mejorar y hacer accesible a todos la movilidad urbana, así como a reducir la contaminación dentro de las grandes ciudades.\\

Por esta razón, en el presente Trabajo de Fin de Grado hemos realizado un análisis de los datos correspondientes a los tiempos de llegada de los autobuses urbanos de la EMT, con el fin de mejorar su rendimiento y fiabilidad.\\

Con esta finalidad, hemos desarrollado una serie de algoritmos para estimar datos derivados de los datos originales, tales como la posición, los tiempos de llegada, el tiempo de recorrido entre las paradas y los intervalos de tiempo entre autobuses consecutivos, y hemos caracterizado y analizado su comportamiento utilizando herramientas de inferencia estadística.\\

Haciendo uso de los datos derivados, junto con herramientas de modelado estadístico, hemos diseñado y probado un modelo de detección de anomalías en tiempo real no supervisado sobre los intervalos de tiempo entre autobuses consecutivos. Las anomalías detectadas por este modelo constituyen una valiosa retroalimentación sobre la calidad del servicio prestado por los autobuses, que puede utilizarse para tomar las decisiones oportunas orientadas a mejorar dicho servicio.\\

Por último, hemos desarrollado una herramienta de visualización web, con el fin de supervisar la detección de anomalías en tiempo real y facilitar la comprensión de las herramientas desarrolladas.\\ \\

\textbf{Palabras clave:} transporte público, movilidad urbana, análisis de datos, tratamiento de datos, ciencia de los datos, autobuses, estadística, intervalos de tiempo entre autobuses, aprendizaje de máquina no supervisado, inferencia estadística, anomalías, visualización. 






